\documentclass[12pt]{article}

% This first part of the file is called the PREAMBLE. It includes
% customizations and command definitions. The preamble is everything
% between \documentclass and \begin{document}.

\usepackage[letterspace=50]{microtype} %change tracking
\usepackage[margin=1in]{geometry}  % set the margins to 1in on all sides
\usepackage{graphicx}              % to include figures
\usepackage{subfig}				   % sublfloat
\usepackage[bottom]{footmisc}      % for footnotes
\usepackage{wrapfig}			   % wrap text around figure
\usepackage[space]{grffile}		   % for spaces in directories
\usepackage{amsmath}               % great math stuff
\usepackage{amsfonts}              % for blackboard bold, etc
\usepackage{amsthm}                % better theorem environments
\usepackage[figuresright]{rotating}			   % for rotation of figures
\usepackage[usenames,dvipsnames]{xcolor}			   % for colored text
\usepackage{libertine}			   % linux libertine
\usepackage[T1]{fontenc}
\usepackage[scaled=0.85]{zi4}	   %incosolata for monospace
\usepackage{verbatim}			   % for verbatim txt file
\usepackage{csvsimple}			   % for csv file input
%\usepackage[scaled]{beramono}     %beramono for monospace
%\usepackage[libertine]{newtxmath}
%\usepackage{pxfonts}			   % palatino font package
%\usepackage{fouriernc}            % fourier font package
\usepackage{changepage}   		   % for the adjustwidth environment
\renewcommand{\baselinestretch}{1.15} % line spacing`
\setlength{\parindent}{2em}		   % paragraph indent
\setlength{\parskip}{0.75em}	   % paragraph skip
%\usepackage{natbib}			   % citation
\usepackage{hyperref}			   % for hyperlinks
\usepackage[super]{nth}            % for superscripts
\usepackage{nth}
\usepackage{titlesec}


\begin{document}
	
\title{\textbf{COS 222 Spring 2015 Report: Noms}}
	
\author{
		Zi Xiang Pan \\
		Princeton University}
	
\end{document}	
	